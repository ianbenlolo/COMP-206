% ***********************************************************
% ******************* PHYSICS HEADER ************************
% ***********************************************************
% Version 2
\documentclass[11pt]{article} 
\usepackage{amsmath} % AMS Math Package
\usepackage{amsthm} % Theorem Formatting
\usepackage{amssymb}	% Math symbols such as \mathbb
\usepackage{graphicx} % Allows for eps images
\usepackage{multicol} % Allows for multiple columns
\usepackage[dvips,letterpaper,margin=0.75in,bottom=0.5in]{geometry}
 % Sets margins and page size
\pagestyle{empty} % Removes page numbers
\makeatletter % Need for anything that contains an @ command 
\renewcommand{\maketitle} % Redefine maketitle to conserve space
{ \begingroup \vskip 10pt \begin{center} \large {\bf \@title}
	\vskip 10pt \large \@author \hskip 20pt \@date \end{center}
  \vskip 10pt \endgroup \setcounter{footnote}{0} }
\makeatother % End of region containing @ commands
\renewcommand{\labelenumi}{(\alph{enumi})} % Use letters for enumerate
% \DeclareMathOperator{\Sample}{Sample}
\let\vaccent=\v % rename builtin command \v{} to \vaccent{}
\renewcommand{\v}[1]{\ensuremath{\mathbf{#1}}} % for vectors
\newcommand{\gv}[1]{\ensuremath{\mbox{\boldmath$ #1 $}}} 
% for vectors of Greek letters
\newcommand{\uv}[1]{\ensuremath{\mathbf{\hat{#1}}}} % for unit vector
\newcommand{\abs}[1]{\left| #1 \right|} % for absolute value
\newcommand{\avg}[1]{\left< #1 \right>} % for average
\let\underdot=\d % rename builtin command \d{} to \underdot{}
\renewcommand{\d}[2]{\frac{d #1}{d #2}} % for derivatives
\newcommand{\dd}[2]{\frac{d^2 #1}{d #2^2}} % for double derivatives
\newcommand{\pd}[2]{\frac{\partial #1}{\partial #2}} 
% for partial derivatives
\newcommand{\pdd}[2]{\frac{\partial^2 #1}{\partial #2^2}} 
% for double partial derivatives
\newcommand{\pdc}[3]{\left( \frac{\partial #1}{\partial #2}
 \right)_{#3}} % for thermodynamic partial derivatives
\newcommand{\ket}[1]{\left| #1 \right>} % for Dirac bras
\newcommand{\bra}[1]{\left< #1 \right|} % for Dirac kets
\newcommand{\braket}[2]{\left< #1 \vphantom{#2} \right|
 \left. #2 \vphantom{#1} \right>} % for Dirac brackets
\newcommand{\matrixel}[3]{\left< #1 \vphantom{#2#3} \right|
 #2 \left| #3 \vphantom{#1#2} \right>} % for Dirac matrix elements
\newcommand{\grad}[1]{\gv{\nabla} #1} % for gradient
\let\divsymb=\div % rename builtin command \div to \divsymb
\renewcommand{\div}[1]{\gv{\nabla} \cdot #1} % for divergence
\newcommand{\curl}[1]{\gv{\nabla} \times #1} % for curl
\let\baraccent=\= % rename builtin command \= to \baraccent
\renewcommand{\=}[1]{\stackrel{#1}{=}} % for putting numbers above =
\newtheorem{prop}{Proposition}
\newtheorem{thm}{Theorem}[section]
\newtheorem{lem}[thm]{Lemma}
\theoremstyle{definition}
\newtheorem{dfn}{Definition}
\theoremstyle{remark}
\newtheorem*{rmk}{Remark}

% ***********************************************************
% ********************** END HEADER *************************
% ***********************************************************
\usepackage{cancel}
\usepackage{mathtools}
\usepackage{amsmath}
\usepackage{pdfpages}
\usepackage{appendix}
\usepackage{enumerate}
\title{Comp 350 Assignment 3: Solving linear systems}
\author{Ian Benlolo 260744397\\McGill University \\}
\begin{document}
\maketitle

\begin{enumerate}[1)]
\item \begin{enumerate}[(a)]
	\item
\[
\begin{bmatrix}
-6 & -4 & 46 & 32 & \vline &-6 \\
12 & 24 & -12 & -24 & \vline & 0\\
6 & 36 & 6 & 12 & \vline & 18\\
6 & 4 & -1 & 31 & \vline & 15\\
\end{bmatrix}
\xrightarrow[]{P_1 }
\begin{bmatrix}
12 & 24 & -12 & -24 & \vline & 0\\
-6 & -4 & 46 & 32 & \vline &-6 \\
6 & 36 & 6 & 12 & \vline & 18\\
6 & 4 & -1 & 31 & \vline & 15\\
\end{bmatrix}
\xrightarrow[R_3 \& R_4 -\frac{1}{2}R_1]{R_2+\frac{1}{2}R_1 }
\begin{bmatrix}
12 & 24 & -12 & -24 & \vline & 0\\
0 & 8 & 40 & 20 & \vline & -6\\ 
0 & 24 & 12 & 24 & \vline & 18\\
0 & -8 & 5 & 43 & \vline & 15\\
\end{bmatrix}
\]
\[
\xrightarrow[]{P_2}
\begin{bmatrix}
12 & 24 & -12 & -24 & \vline & 0\\
0 & 24 & 12 & 24 & \vline & 18\\
0 & 8 & 40 & 20 & \vline & -6\\ 
0 & -8 & 5 & 43 & \vline & 15\\
\end{bmatrix}
\xrightarrow[R_3 - \frac{1}{3}R_2]{ R_4 +  \frac{1}{3}R_2}
\begin{bmatrix}
12 & 24 & -12 & -24 & \vline & 0\\
0 & 24 & 12 & 24 & \vline & 18\\
0 & 0 & 36 & 12 & \vline & -12\\
0 & 0 & 9 & 51 & \vline & 21\\
\end{bmatrix}
\xrightarrow[R_4-\frac{1}{4}R_3]{}
\]
\[
\begin{bmatrix}
12 & 24 & -12 & -24 & \vline & 0\\
0 & 24 & 12 & 24 & \vline & 18\\
0 & 0 & 36 & 12 & \vline & -12\\
0 & 0 & 0 & 48 & \vline & 24 \\
\end{bmatrix}
\]
Where, 
\[
P_1
=
\begin{bmatrix}
0 & 1 & 0& 0\\
1 & 0 & 0 & 0\\
0 & 0 & 1 & 0\\
0 & 0 & 0 & 1\\
\end{bmatrix}
, P_2=
\begin{bmatrix}
1 & 0 & 0 & 0\\
0 & 0 & 1 & 0\\
0 & 1 & 0& 0\\
0 & 0 & 0 & 1\\
\end{bmatrix}
\]
Now, using back substitution we find that $$ x_4=\frac{1}{2}, x_3=\frac{-1}{2},  x_2=\frac{1}{2}, x_1=\frac{-1}{2}$$
	\item 
	\[
	L=
	\begin{bmatrix}
	1 & 0 & 0 & 0\\
	\frac{1}{2} & 1 & 0 & 0\\
	\frac{-1}{2} & \frac{1}{3} & 1 & 0\\
	\frac{1}{2} & \frac{-1}{3} & \frac{1}{4}& 1\\
 	\end{bmatrix}
	, U=\begin{bmatrix}
12 & 24 & -12 & -24  \\
0 & 24 & 12 & 24 \\
0 & 0 & 36 & 12 \\
0 & 0 & 0 & 48  \\
\end{bmatrix}
\]
\[
PA=LU=
	\begin{bmatrix}
	1 & 0 & 0& 0\\
0 & 0 & 1 & 0\\
0 &1 & 0 & 0\\
0 & 0 & 0 & 1\\
	\end{bmatrix}
	\begin{bmatrix}
	0 & 1 & 0& 0\\
1 & 0 & 0 & 0\\
0 & 0 & 1 & 0\\
0 & 0 & 0 & 1\\
	\end{bmatrix}
	\times A
	=
	\begin{bmatrix}
	1 & 0 & 0 & 0\\
	\frac{1}{2} & 1 & 0 & 0\\
	\frac{-1}{2} & \frac{1}{3} & 1 & 0\\
	\frac{1}{2} & \frac{1}{3} & \frac{1}{4}& 1\\
 	\end{bmatrix}
	\begin{bmatrix}
12 & 24 & -12 & -24  \\
0 & 24 & 12 & 24 \\
0 & 0 & 36 & 12 \\
0 & 0 & 0 & 48  \\
\end{bmatrix}
	\]
	\[
	A=
	\begin{bmatrix}
	0 & 1 & 0 & 0 \\
	1 & 0 & 0 & 0  \\
	0 & 0 & 1 & 0\\
	0 & 0 & 0 & 1\\
 	\end{bmatrix}
	\begin{bmatrix}
	1 & 0 & 0 & 0  \\
	0 & 0 & 1 & 0\\
	0 & 1 & 0 & 0 \\
	0 & 0 & 0 & 1\\
 	\end{bmatrix}
	\begin{bmatrix}
	1 & 0 & 0 & 0\\
	\frac{1}{2} & 1 & 0 & 0\\
	\frac{-1}{2} & \frac{1}{3} & 1 & 0\\
	\frac{1}{2} & \frac{1}{3} & \frac{1}{4}& 1\\
 	\end{bmatrix}
	\begin{bmatrix}
12 & 24 & -12 & -24  \\
0 & 24 & 12 & 24 \\
0 & 0 & 36 & 12 \\
0 & 0 & 0 & 48  \\
\end{bmatrix}
=
\begin{bmatrix}
-6 & -4 & 46 & 32  \\
12 & 24 & -12 & -24 \\
6 & 36 & 6 & 12 \\
6 & 4 & -1 & 31 \\
\end{bmatrix}
	\]
	Which does in fact give the original matrix!
	\end{enumerate}

\item
	\begin{enumerate}[(a)]
	\item
	\begin{verbatim}
	function x =genp(A,b)
% input (nonsingular) n by n matrix A and n by 1 matrix b
%output solution to Ax=b
%A is nxn so n^2 memory; b is nx1 for n memory

n=length(b);
midpoint = (n+1)/2 ; %2 flops, 1 memory

for k=1:n-1 %1 memory
    if k==midpoint % skip the middle point because we'd get a row of 0's if not, 1 flop
        continue;
    else
        i=k+1:n; % 1 memory n-k-1 flops
        A(i,k) =A(i,k)/A(k,k); % 1 flop
        A(i,i)=A(i,i)-A(i,k)*A(k,i); % 2 flops
        b(i)=b(i)-A(i,k)*b(k); % 2 flops
    end
end

%back substitution
x=zeros(n,1); % 1 flop, n memory
x(n)=b(n)/A(n,n); % 1 flop
for k=n-1:-1:1 
    x(k)=(b(k)-A(k,k+1:n)*x(k+1:n))/A(k,k);  % 5 flops
end

end
	\end{verbatim}
 This all totals out to 
 $1+1+\sum\limits_{i=1}^{n-1}(6+n-i-1)+1+1+\sum\limits_{i=1}^{n-1}(5)=\dots=\frac{n^2}{2}-\frac{7n}{2}-1$\\
 Memory allocation: 
 $n^2+n+n+2$

	\begin{verbatim}
function x = gepp(A,b)
%input: A is an n x n nonsingular matrix
%       b is an n x 1 vector
% output: x is the solution of Ax=b
%A is nxn so n^2 memory; b is nx1 for n memory
n = length(b); %1 flop

midpoint = (n+1)/2; % n is odd so this is a whole number, 2 flops
for k = 1:n-1 % 1 memory
    [maxval, maxindex] = max(abs(A(k:n,k))); % 2 flops, 2 memory
    if maxval==0, error('A is singular'), end % 1 flop
    
    q = maxindex+k-1; % 2 flops
    A([k,q],k:n) = A([q,k],k:n); 
   
    b([k,q]) = b([q,k]);
    if k == midpoint % 1 flop
        continue;
    else
        i = k+1:n; % n -k-1 flops
        A(i,k) = A(i,k)/A(k,k); % 1 flop
        A(i,i) = A(i,i) - A(i,k)*A(k,i); % 2 flops
        b(i) = b(i) - A(i,k)*b(k); % 2 flops
    end
end
x = zeros(n,1); %%backwards substitution, n memory
x(n) = b(n)/A(n,n); % 1 flop
for k = n-1:-1:1 % 1 memory
    x(k) = (b(k) - A(k,k+1:n)*x(k+1:n))/A(k,k);% 4 flops
end
	\end{verbatim}
Total flops:
$1+2+1+\sum\limits_{i=1}^{n-1}(2+1+2+n-i-1+1+4)+n+1\sum\limits_{i=1}^{n-1}(4)=\dots=\frac{3n^2}{2}+\frac{23n}{2}-22$\\
Total memory allocation: $n^2+n+n+2=n^2+2n+2$
	Though my algorithms worked well, before handing in this assignment i realized (for what it's worth) that there could have been a better way to implement this algorithm which would be much less memory-heavy. This algorithm would have taken in as input $3$ arrays; one for each diagonal and another for b. Sadly, I figured this out with not enough time to spare in actually coding it all the way through so i decided to simply hand it what I had!

	\item, (c), (d): see pdf at bottom for output. \\
	
	It is worth it to note that Xnp is much larger in questions (c) and (d) than in (b). This is because in GENP, the algorithm is dividing many large numbers by a very small one which introduces a lot of error. This error is introduced because of the limitations of floating point numbers. This also then introduced a small error in computing the solution matrix in (c) and (d). 
The error for GEPP is quite consistently very small. 
		
	\begin{verbatim}
	function testscript()

n=9;
diagonal1 = randn(1, 2*n+1);
diagonal2 = randn(1, 2*n+1);

matrix1 = zeros(2*n+1, 2*n+1);
matrix1=diag(diagonal1);
A=fliplr(matrix1);
A=A+diag(diagonal2); % Random A is done
b=zeros(2*n+1,1);

midpt=n+1;
%since x = one's, we wnat to add all the horizontal entries in Ai into bi

for i=1:2*n+1
    if i==midpt
        b(i,1)=A(i,i);
    else
        b(i,1)=A(i,i)+A(i,2*n+2-i);
    end
end
% at this point we have A and b such that the solution to Ax=b is a matrix of ones

%%now well test my genp and gepp programs

Xnp=genp(A,b); 
Xpp=gepp(A,b); 
x=ones(2*n+1,1);
epsilon=eps*cond(A,2); % this calculates epsilon*norm(A)*norm(A^-1)

disp("-----------------------------------------------------------------")
XnpErr = (norm((x-Xnp),2))/(norm(x,2)); %errors for GEPP and GENP
XppErr = (norm((x-Xpp),2))/(norm(x,2)); % these should be 0 but aren't due 
%to discretization error of the computer!

disp("Data: ")
disp(" A= ")
disp(A)
disp("b = ")
disp(b)

disp("Result for ")
disp("     GENP      GEPP")
disp([Xnp, Xpp])

disp("Error in Xnp= "+ XnpErr)
disp("Error in Xpp= "+ XppErr)
disp("Epsilon*||A||*||A^1||= "+epsilon)

disp("")
disp("*****************************************************************")
disp("")

disp("Question 2.c)")

A(1,1)=10e-15;
b(1,1)=A(1,1)+A(1,2*n+2-1); %% made the changes required by the question

disp("Data: ")
disp(" A= ")
disp(A)
disp("b = ")
disp(b)

Xnp=genp(A,b);  %%now we calculate and print everything as before
Xpp=gepp(A,b); 
epsilon=eps*cond(A,2); 
XnpErr = (norm((x-Xnp),2))/(norm(x,2)); %errors for GEPP and GENP
XppErr = (norm((x-Xpp),2))/(norm(x,2)); % these should be 0 but aren't due to 
%discretization error of the computer!


disp("Result for ")
disp("     GENP      GEPP")
disp([Xnp, Xpp])

disp("Xnp error: "+ XnpErr)
disp("Xpp error: "+ XppErr)
disp("Epsilon*||A||*||A^1||= "+epsilon)

disp("*****************************************************************") 

disp("Question 2.d)")
A(2*n+1,1)=10e-8;
b(2*n+1,1)=A(2*n+1,1)+A(2*n+1,2*n+2-1); %% made the changes 
%required by the question
%now we calculate and print everything as before

disp("Data: ")
disp(" A= ")
disp(A)
disp("b = ")
disp(b)

Xnp=genp(A,b);  
Xpp=gepp(A,b); 
epsilon=eps*cond(A,2); 

XnpErr = (norm((x-Xnp),2))/(norm(x,2)); %errors for GEPP and GENP
XppErr = (norm((x-Xpp),2))/(norm(x,2)); % these should be 0 but aren't 
%due to discretization error of the computer!

disp("Result for ")
disp("     GENP      GEPP")
disp([Xnp, Xpp])
disp("Xnp error: "+ XnpErr)
disp("Xpp error: "+ XppErr)
disp("Epsilon*||A||*||A^1||= "+epsilon)


	\end{verbatim}
	
	\end{enumerate}
	
	

\end{enumerate}

\includepdf[pages=-6]{testscript.pdf}

\end{document}